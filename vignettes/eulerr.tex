%\VignetteIndexEntry{eulerr under the hood}

\PassOptionsToPackage{usenames,fixpdftex,dvipsnames,svgnames,x11names}{xcolor}
\PassOptionsToPackage{hyphens}{url}
\PassOptionsToPackage{font=small,labelfont=bf,labelsep=period}{caption}

\documentclass[
  oneside,
  openany,
  numbers=noendperiod,
  parskip=half,
  bibliography=totoc
]{scrbook}\usepackage[]{graphicx}\usepackage{xcolor}
%% maxwidth is the original width if it is less than linewidth
%% otherwise use linewidth (to make sure the graphics do not exceed the margin)
\makeatletter
\def\maxwidth{ %
  \ifdim\Gin@nat@width>\linewidth
    \linewidth
  \else
    \Gin@nat@width
  \fi
}
\makeatother

\definecolor{fgcolor}{rgb}{0.345, 0.345, 0.345}
\newcommand{\hlnum}[1]{\textcolor[rgb]{0.686,0.059,0.569}{#1}}%
\newcommand{\hlstr}[1]{\textcolor[rgb]{0.192,0.494,0.8}{#1}}%
\newcommand{\hlcom}[1]{\textcolor[rgb]{0.678,0.584,0.686}{\textit{#1}}}%
\newcommand{\hlopt}[1]{\textcolor[rgb]{0,0,0}{#1}}%
\newcommand{\hlstd}[1]{\textcolor[rgb]{0.345,0.345,0.345}{#1}}%
\newcommand{\hlkwa}[1]{\textcolor[rgb]{0.161,0.373,0.58}{\textbf{#1}}}%
\newcommand{\hlkwb}[1]{\textcolor[rgb]{0.69,0.353,0.396}{#1}}%
\newcommand{\hlkwc}[1]{\textcolor[rgb]{0.333,0.667,0.333}{#1}}%
\newcommand{\hlkwd}[1]{\textcolor[rgb]{0.737,0.353,0.396}{\textbf{#1}}}%
\let\hlipl\hlkwb

\usepackage{framed}
\makeatletter
\newenvironment{kframe}{%
 \def\at@end@of@kframe{}%
 \ifinner\ifhmode%
  \def\at@end@of@kframe{\end{minipage}}%
  \begin{minipage}{\columnwidth}%
 \fi\fi%
 \def\FrameCommand##1{\hskip\@totalleftmargin \hskip-\fboxsep
 \colorbox{shadecolor}{##1}\hskip-\fboxsep
     % There is no \\@totalrightmargin, so:
     \hskip-\linewidth \hskip-\@totalleftmargin \hskip\columnwidth}%
 \MakeFramed {\advance\hsize-\width
   \@totalleftmargin\z@ \linewidth\hsize
   \@setminipage}}%
 {\par\unskip\endMakeFramed%
 \at@end@of@kframe}
\makeatother

\definecolor{shadecolor}{rgb}{.97, .97, .97}
\definecolor{messagecolor}{rgb}{0, 0, 0}
\definecolor{warningcolor}{rgb}{1, 0, 1}
\definecolor{errorcolor}{rgb}{1, 0, 0}
\newenvironment{knitrout}{}{} % an empty environment to be redefined in TeX

\usepackage{alltt}

% Page layout
\usepackage[
  a4paper,
  left=23mm,
  top=27.4mm,
  bottom=27.4mm,
  %headsep=2\baselineskip,
  textwidth=107mm,
  marginparsep=8mm,
  marginparwidth=49mm,
  %textheight=49\baselineskip,
  %headheight=\baselineskip
]{geometry}
%\usepackage{showframe}
\usepackage{multicol}

% Margin paragraph
\usepackage{sidenotes}
\usepackage{marginfix}
%\usepackage{morefloats} % More than 18 floats

% Headers and footers
\usepackage[
  automark,
  headsepline=false,
  headwidth=textwithmarginpar,
  footwidth=head
]{scrlayer-scrpage}

\clearpairofpagestyles
\rofoot[\pagemark]{\pagemark}
\lohead{\headmark}

% Title page
\title{eulerr: Area-Proportional Euler Diagrams with Ellipses}
\author{Johan larsson}
\date{\today}

\renewcommand{\maketitle}{%
  \cleardoublepage
  \begin{fullwidth}
  \centering
  \vspace*{3cm}
  \includegraphics[width=0.3\textwidth]{LundUniversity_C2line_BLACK}\par\vspace{1cm}
  \vspace{0.5cm}
  {\scshape\Large Bachelor Thesis \par}
  {\Huge\bfseries eulerr: Area-Proportional Euler Diagrams with Ellipses \par}
  \vspace{2cm}
  {\huge\itshape Johan Larsson \par}
  \vspace{2cm}
  {\Large{\itshape supervised by}\par Peter Gustafsson}
  \vfill
  {\large \today\par}
  \end{fullwidth}
  \thispagestyle{empty}
}

% Graphics
\usepackage[font=footnotesize]{subcaption}
\usepackage{graphicx}
\setkeys{Gin}{width=\linewidth,totalheight=\textheight,keepaspectratio}
\graphicspath{{graphics/}}
%\captionsetup[sub]{font = footnotesize}

\usepackage{booktabs}
%\usepackage[header,page]{appendix}

% Text justification and parskip
\usepackage[document]{ragged2e}
%\setlength{\RaggedRightParindent}{\parindent}

% Lists
\usepackage[shortlabels]{enumitem}
%\setlist{listparindent=\parindent, parsep=0pt plus 1pt}

% Algorithms
\usepackage[vlined]{algorithm2e}

% Babel
%\usepackage[swedish]{babel}

% Bibliography
\usepackage[square,numbers,sort&compress]{natbib}
\usepackage{etoolbox}
%\usepackage{relsize}
\patchcmd{\thebibliography}
  {\list}
  {\begin{multicols}{2}\small\list}
  {}
  {}
\appto{\endthebibliography}{\end{multicols}}

% Floats
% \usepackage{newfloat}
% \DeclareFloatingEnvironment[fileext=alg,placement={htbp},name=Algorithm]{alg}
% \captionsetup[myfloat]{labelfont=bf,labelsep=period,font=small}
\DeclareCaptionType[fileext=alg,name=Algorithm]{alg}

% Extended verbatim environments
\usepackage{fancyvrb}
\fvset{fontsize=\small}% default font size for fancy-verbatim environments

% Provide fullwidth environment
\newlength{\overhang}
\setlength{\overhang}{\marginparwidth}
\addtolength{\overhang}{\marginparsep}
\makeatother

\newenvironment{fullwidth}{
  \begin{addmargin*}[0em]{-\overhang}
  }{
  \end{addmargin*}
}

% Fonts
\usepackage[utf8]{inputenc}
\usepackage[T1]{fontenc}
\usepackage[lining]{libertine}
\usepackage{textcomp}
\usepackage[varqu,varl,scaled=0.93]{inconsolata}
\usepackage{mathtools}
\usepackage{amsthm}
\usepackage[libertine,vvarbb,libaltvw,liby]{newtxmath}
\usepackage[scr=rsfso]{mathalfa}
\usepackage{bm}
\useosf
\usepackage{microtype}

\newcommand{\proglang}[1]{\textsf{#1}}
\newcommand{\pkg}[1]{{\fontseries{b}\selectfont #1}}
\newcommand{\code}[1]{\texttt{#1}}

% Custom operators
\DeclareMathOperator{\E}{E}
\DeclareMathOperator{\Pois}{Pois}
\DeclareMathOperator{\B}{Bin}
\DeclareMathOperator{\V}{Var}
\DeclareMathOperator{\Exp}{Exp}

% Caption styles (caption package is loaded with sidenotes)
\DeclareCaptionStyle{sidecaption}{labelfont=bf,labelsep=period,font=small}
\DeclareCaptionStyle{marginfigure}{labelfont=bf,labelsep=period,font=small}
\DeclareCaptionStyle{margintable}{labelfont=bf,labelsep=period,font=small}
\DeclareCaptionStyle{widefigure}{labelfont=bf,labelsep=period,font=small}
\DeclareCaptionStyle{widetable}{labelfont=bf,labelsep=period,font=small}

% Cross-referencing and colors
\usepackage{hyperref}
\usepackage[noabbrev,capitalize,nameinlink]{cleveref}
\usepackage{xcolor}
\hypersetup{linkcolor=SteelBlue4,
            citecolor=SteelBlue4,
            urlcolor=SteelBlue4,
            colorlinks=true}

% Theorems
\newtheorem{mydef}{Definition}[chapter]

% Custom title page


% Change fontsize in sidenotes
\makeatletter
\ExplSyntaxOn
\RenewDocumentCommand \sidenotetext { o o +m }
{
  \IfNoValueOrEmptyTF{#1}
    {
      \@sidenotes@placemarginal{#2}{\textsuperscript{\thesidenote}{}\small~#3}
  \refstepcounter{sidenote}
}
    {\@sidenotes@placemarginal{#2}{\textsuperscript{#1}\small~#3}}
}
\ExplSyntaxOff
\makeatother
\IfFileExists{upquote.sty}{\usepackage{upquote}}{}
\begin{document}



\frontmatter

\maketitle

\begin{addmargin*}[0.5\overhang]{-0.5\overhang}
{\hypersetup{linkcolor=black}
\tableofcontents
}
\end{addmargin*}
\mainmatter

\chapter{Background}\label{sec:background}

The visual display of data represents an intuitive form of data presentation.
Data visualizations work on multiple dimensions and possess the
potential to convey intricate relationships that single statistics or tables
never can.

Such visualizations, however, are only effective if their aesthetics convey
relationships. Consider, for instance, a disc with a radius of 2~cm labelled
\emph{Men}\marginpar{%
\begin{knitrout}\small
\definecolor{shadecolor}{rgb}{0.969, 0.969, 0.969}\color{fgcolor}

{\centering \includegraphics[width=\maxwidth]{figure/graphics-men-1} 

}



\end{knitrout}
}---it says nothing by itself; yet if we juxtapose it with a 1~cm-radius disc
labelled \emph{Children}\marginpar{%
\begin{knitrout}\small
\definecolor{shadecolor}{rgb}{0.969, 0.969, 0.969}\color{fgcolor}

{\centering \includegraphics[width=\maxwidth]{figure/graphics-children-1} 

}



\end{knitrout}
}, the graphic starts to become informative: it now displays the relation
between two quantities. Now, if we intersect the two
discs, so as to produce an overlap, we have successfully visualized the
relative proportions of men and men, as well as their intersection. The diagram
we have constructed is a \emph{Euler diagram}~(\cref{fig:children-men}).

\begin{marginfigure}
\begin{knitrout}\small
\definecolor{shadecolor}{rgb}{0.969, 0.969, 0.969}\color{fgcolor}

{\centering \includegraphics[width=\maxwidth]{figure/graphics-childrenMen-1} 

}



\end{knitrout}
\caption{The merits of a Euler diagram.}
\label{fig:children-men}
\end{marginfigure}

The Euler diagram, originally proposed by Leonard Euler~\citep{euler_1802}, is
a superset of the obiquiteous \emph{Venn diagram}: a staple of introductory
text books in statistics and research disciplines such as biomedicine and
geology. Venn and Euler diagrams differ in that the the former require all
intersections to be present---even if they are empty---whilst Euler diagrams do
not.

Euler diagrams may moreover be area-proportional, which is to say that each separate
surface of the diagram maps to some quantity. (This was the case with
the diagram with defined in the second paragraph.) This is a rational form for a
Euler diagram---only its geometries are necessary to interpret it, letting us, for
instance, to discard numbers without crucial loss of information; the same
cannot be said for a Venn diagram.

Area-proportional Euler diagrams may be fashioned out of any closed shape, and
have been implemented for triangles~\citep{swinton_2011},
rectangles~\citep{swinton_2011}, ellipses~\citep{micallef_2014}, smooth
curves~\citep{micallef_2014b}, polygons~\citep{swinton_2011}, and
circles~\citep{wilkinson_2012,kestler_2008,swinton_2011}. Circles is the
popular choice, and for good reason, since they are easiest to
interpret~\citep{blake_2016}. In spite of this, circles do not always lend
themselves to accurate representations. Consider, for instance the following
three-set relationship:
\[
\begin{gathered}
A = B = C = 2,\\
A \cap B = A \cap C = B \cap C = 1\\
A \cap B \cap C = 0.
\end{gathered}
\]
There is no way to visualize this relationship perfectly with circles because
they cannot be arranged so that the $A \cap B \cap C$ overlap remains empty whilst
$A \cap B$, $A \cap C$, and $B \cap C$ are non-empty. With ellipses, however,
we can solve this problem since they can both be stretched and rotated, enabling
a perfect fit~(\cref{fig:impossible}). In essence, circles
feature three degrees of freedom: a center consisting of x- and
y-coordinates $h$ and $k$, as well as a radius $r$. Ellipses, meanwhile, have
five: the aforementioned $h$ and $k$, a semi-major axis $a$, a semi-minor axis
$b$, and an angle of rotation $\phi$.

\begin{marginfigure}
\begin{knitrout}\small
\definecolor{shadecolor}{rgb}{0.969, 0.969, 0.969}\color{fgcolor}

{\centering \includegraphics[width=\maxwidth]{figure/graphics-impossible-1} 
\includegraphics[width=\maxwidth]{figure/graphics-impossible-2} 

}



\end{knitrout}
\caption{A set relationship depicted erroneously with circles and perfectly with
  ellipses.}
\label{fig:impossible}
\end{marginfigure}

With four or more intersecting sets, exact circular Euler diagrams are in fact
impossible, given that we \emph{require} 15 intersections but with four circles
can yield at most 13 unique overlaps. This is not the case with ellipses, which
may intersect in up to four, rather than two, points. The only implementation of
elliptical Euler diagrams is found in \pkg{eulerAPE}~\citep{micallef_2014}, yet
it only supports three sets that are moreover required to intersect. The
diagram in~\cref{fig:eulerape-no}, for instance, would not be possible
with \pkg{eulerAPE}.

\begin{marginfigure}
\begin{knitrout}\small
\definecolor{shadecolor}{rgb}{0.969, 0.969, 0.969}\color{fgcolor}

{\centering \includegraphics[width=\maxwidth]{figure/graphics-unnamed-chunk-1-1} 

}



\end{knitrout}
\caption{A Euler diagram with a subset relationship.}
\label{fig:eulerape-no}
\end{marginfigure}

Euler diagrams do not reduce to analytical solutions~\citep{chow_2007} and have
to be solved numerically. Most implementations accomplish this in two steps,
first finding a rough initial estimate that is then finalized in a second, more
accurate, algorithm. For the initial configuration,
\pkg{eulerAPE}~\citep{micallef_2013}, for instance, uses a greedy algorithm that
tries to minimze the error in the three-way intersection.
\pkg{venneuler}~\citep{wilkinson_2012} uses multi-dimensional scaling with
Jacobian distances, taking only pairwise relationships into account.
\pkg{venn.js}~\citet{frederickson_2016} uses a constrained version of the latter
that is instead based on euclidean distances and separately runs a greedy
algorithm, picking the best fit out of the two.
\pkg{Vennerable}~\citet{swinton_2011} uses a simple method of computing the
required pairwise distances between circles and the changes the largest to
attempt to arrive at the correct two-way overlaps. All the algorithms use
circles in the initial configuration.

Diagrams with more than two sets normally require additional tuning, which may
be dealt with in a final configuration step. The prerequisite for this is that
we first compute the areas of the overlaps in order to establish how well our
diagram fits the input. Calculating overlaps, however, is no trivial task,
particularly not for ellipses. This is evident in that most methods resort to
approximations such as quad-tree binning~\citep{wilkinson_2012}, polygon
intersecting~\citep{kestler_2008}, or restricting the algorithm to pairwise
overlaps~\citep{swinton_2011}. \citet{frederickson_2016}, contrastingly,
computes areas exactly, yet only for circles. These three approaches moreover allow only
circles, whilst \citet{micallef_2013}, on the other hand, compute areas exactly
also for ellipses, though only for a maximum of three.

Compared to approximative methods, exact algorithms require that we
first find all the intersection points between the ellipses, for which there are
several approaches. Some rely on solving the system of equations formed by two
ellipses to be intersected, which necessitates solving a fourth-degree
polynomial; other methods make use of the representation of an ellipse as a
conic in projective geometry, which involves solving a third-degree polynomial.
These methods vary in execution time but are all accurate up to
floating-point precision.

With all the intersection points at hand, the areas of the overlaps can be
established. \citet{frederickson_2013} has published a method for circles
and a similar method is used by \citet{micallef_2013} for three ellipses. No
method has so far been published that
generalizes these methods to diagrams of more than three ellipses or ellipses
with subset or disjoint relationships.

\section{Aims}
\label{sec:aims}

Elliptical Euler diagrams have previously not been implemented for more than
three sets or for three-set diagrams with subset or disjoint relationships. This
is the motivation for this thesis, with which we aim to present
a method and implementation for constructing and visualizing Euler diagrams for
sets of any numbers using ellipses and exact area computations.

\chapter{Method}
\label{ch:method}

Constructing a Euler diagram is analagous to fitting a statistical model in that
you need
\begin{enumerate}
\item data,
\item a model to fit the data on,
\item a test to assess the model fit, and
\item a presentation of the result.
\end{enumerate}
In the following sections, we explain how \pkg{eulerr} adressess each of these
items in turn.

\section{Input}
\label{sec:input}

The data for a Euler diagram is always a description of set relationships.
\pkg{eulerr} allows several alternatives for this data, namely,
\begin{itemize}
\item intersections and relative complements\sidenote{%
    $A \setminus B = 3 \quad B \setminus A = 2 \quad A \cap B=1$
  },
\item unions and identities\sidenote{%
    $A=4 \quad B=3 \quad A \cap B=1$
  },
\item a matrix of binary (or boolean) indices\sidenote{%
    $\begin{bmatrix}
      \bm{A} & \bm{B} & \bm{C}\\
      0 & 1 & 0 \\
      1 & 1 & 1 \\
      1 & 0 & 0 \\
    \end{bmatrix}$
  },
\item a list of sample spaces\sidenote{%
    $\begin{matrix}
      A = \{ab,\,bb,\,bc\}\\
      B = \{aa,\,bc,\,cc\}\\
      C = \{bb,\,bb,\,cc\} \end{matrix}
    $
  }, or
\item a two- or three-way table \sidenote{%
    \begin{tabular}{lrr}
    \toprule Survived? & No  & Yes \\
    \midrule Child & 52.00 & 57.00 \\
    Adult & 1438.00 & 654.00 \\
    \bottomrule\end{tabular}
  }.
\end{itemize}

As an additional feature for the matrix form, the user may supply a factor
variable with which to split the data set before fitting a Euler diagram to each
split. This is offered as a convenience function for the user since it may be
that the solution is more well-behaved after such a split.

Whichever type of input is provided, \pkg{eulerr} translates it to the first,
\emph{intersections and relative complements}~(\cref{def:omega}), which is the
form used later in the loss functions of the initial and final optimizers.

\begin{mydef}
\label{def:omega}
For a family of \emph{N} sets, $F = F_1, F_2, \dots, F_N$, and their $n=2^N-1$
intersections, we define $\omega$ as the intersections of these sets and their
relative complements, such that
\begin{align*}
  \omega_{1} & = F_1 \setminus \bigcap_{j=2}^N F_j  \\
  \omega_{2} & = (F_2 \cap F_3) \setminus \bigcap_{j=3}^{n} F_j\\
  \omega_{3} & = \bigcap_{i=1}^3 F_i \setminus \bigcap_{j=4}^{N} F_j\\
             & \vdotswithin{=} \\
    \omega_n & = \bigcap_{j=1}^{N}F_j
\end{align*}
with
\[
  \sum_{i = 1}^n \omega_i =  \bigcup_{j=1}^N F_j.
\]
Analogously to $\omega$, and for convenience, we also introduce the $\&$
operator as
\[
  F_j \& F_k = (F_j \cap F_k)\setminus (F_j \cap F_k)^C = \omega_{i},
\]
where $i$ in this instance is the index of the binary identifier of the
intersection between $F_j$ and $F_k$.
\end{mydef}

With the input translated into a useable form, the Euler diagram is fit in two
steps: first, an initial configuration is formed with circles using only the
sets' pairwise relationships. Second, this configuration is fine tuned taking
all $2^N-1$ overlaps into account.

\section{Initial configuration}
\label{sec:initConfig}

For our initial configuration we rely on a constrained version of
multi-dimensional scaling~(MDS) from \pkg{venn.js}~\citep{frederickson_2016},
which is a modification of a method from \pkg{venneuler}~\citep{wilkinson_2012}.
In it, we consider the pairwise relationsships between the sets and attempt to
position their respective shapes so as to minimize the difference between the
distance between their centers required to obtain an optimal overlap ($\omega$)
and the actual overlap between the shapes in the diagram.

This problem is unfortunately intractable for ellipses, being that there is an
infinite number of ways by which we can position two ellipses to obtain a given
overlap. Thus, we restrict ourselves to circles, for which we can use the
circle--circle overlap formula~\eqref{eq:circleOverlap} to numerically find the
required distance, $d$, for each set of two ellipses,
\begin{fullwidth}
\begin{multline}
O_{ij} = r_i^2\arccos\left(\frac{d_{ij}^2 + r_i^2 - r_j^2}{2d_{ij}r_i}\right) +
r_j^2\arccos\left(\frac{d_{ij}^2 + r_j^2 - r_i^2}{2d_{ij}r_j}\right) - \\
\frac{1}{2}\sqrt{(-d_{ij} + r_i + r_j)(d_{ij} + r_i - r_j)(d_{ij} - r_i + r_j)(d_{ij} + r_i + r_j)},
\label{eq:circleOverlap}
\end{multline}
\end{fullwidth}
where $r_i$ and $r_j$ are the radii of the circles representing the $i$:th and
$j$:th sets respectively, $O_{ij}$ their overlap, and $d_{ij}$ the distance
between them.

\begin{marginfigure}
\begin{knitrout}\small
\definecolor{shadecolor}{rgb}{0.969, 0.969, 0.969}\color{fgcolor}

{\centering \includegraphics[width=\maxwidth]{figure/graphics-circleOverlap-1} 

}



\end{knitrout}
\caption{The circle--circle overlap is computed as a function of the discs'
separation ($d_{ij}$), radii ($r_i,r_j$), and area of overlap ($O_{ij}$).}
\label{fig:circleCircle}
\end{marginfigure}

We are looking for $d$, which we find easily from knowing $O$ and $r$. Our loss
function is the squared difference between $O$ and $\omega$
(the desired overlap),
\[
  \mathcal{L}(d_{ij}) = (O_{ij} - \omega_{ij})^2, \quad \text{for } i <
    j \neq < n
\]
which we optimize using R's built-in \code{optimize()}\sidenote{According to the
documentation, \code{optimize()} consists of a "combination of golden section
search and successive parabolic interpolation."}. Convergence is fast and
neglible next to our later optimization procedures.

Given these optimal pairwise distances, we proceed to the next step, where we
position the circles representing the sets. This can be accomplished in many
ways; \pkg{eulerr}

The algorithm assigns a loss and gradient of zero when the sets and their
representations as circles are disjoint or when the sets and circles are subset.
In all other cases, the loss function~\eqref{eq:initLoss} is the normal sums of
squares between the optimal distance between two sets, $d$, that we found
in~\eqref{eq:circleOverlap} and the actual distance in the layout we are
currently exploring. The gradient~\eqref{eq:initGrad} is retrieved as usual by
taking the derivative of the loss function.
\begin{fullwidth}
\begin{align}
\mathcal{L}(\bm{v}) = \sum_{i=1}^{N-1} \sum_{i=j+1}^{N}
& \begin{cases}
  0 & (F_i \cap F_j = \emptyset) \wedge (O_{ij} = \emptyset)\\
  0 & \left((F_i \subseteq F_j) \vee (F_i \supseteq F_j)\right) \wedge (O_{ij}=\emptyset)\\
  4 ((\bm{v}_i - \bm{v}_j)^{\scriptscriptstyle T}(\bm{v}_i - \bm{v}_j) - d_{ij}^2)^2  & \text{otherwise} \\
\end{cases} \label{eq:initLoss} \\
\vec{\nabla} f(\bm{v}_i) = \sum_{i=1}^N
& \begin{cases}
  \vec{0} & (F_i \cap F_j = \emptyset) \wedge (O_{ij} = \emptyset)\\
  \vec{0} & \left((F_i \subseteq F_j) \vee (F_i \supseteq F_j)\right) \wedge (O_{ij}=\emptyset)\\
  4 \left((\bm{v}_i - \bm{v}_j)^{\scriptscriptstyle T}(\bm{v}_i - \bm{v}_j) - d_{ij}^2\right) (\bm{v}_i - \bm{v}_j) & \text{otherwise}, \\
\end{cases} \label{eq:initGrad}
\end{align}
\end{fullwidth}
where $\bm{v}_i = \begin{bsmallmatrix}h_i \\ k_i\end{bsmallmatrix}$ and $\wedge,\vee$ denote the conditional \texttt{AND} and \texttt{OR} operators respectively.

We optimize~\eqref{eq:initLoss} using the nonlinear optimizer \code{nlm()} from
the R core package \pkg{stats}, which is a translation from FORTRAN code
developed by \citet{schnabel_1985} that uses a mix of Newton and Quasi-Newton
algorithms. It makes use of the Hessian, which we presently compute
numerically\sidenote{At the time of writing, there was a bug in the current version of the package,
causing the analytic Hessian to be updated incorrectly.}.

This initial configuration will be accurate for two-set combinations and optimal---although not necessarily accurate---for three-set combinations that use circles. But for all other combinations there is usually a need to fine-tune the configuration.

\section{Final configuration}
\label{sec:finalConfig}

So far, we have only considered pairwise relationships. To test and improve our
layout, however, we need to account for all the relationships and, consequently,
all the intersections and overlaps in the diagram. Initially, we restricted
ourselves to circles but now extend ourselves also to ellipses.

As we saw in the \nameref{sec:background}, we now need to have all the ellipses'
points of intersections at hand. \pkg{eulerr}'s approach to this is outlined in
\citet{richter-gebert_2011} and based in \emph{projective}, as opposed to
\emph{euclidean}, geometry.

To collect all the intersection points, we naturally need only to consider two
ellipses at a time. The canonical form of an ellipse is given by
\[
\frac{\left[ (x-h)\cos{\phi}+(y-k)\sin{\phi} \right]^2}{a^2}+
  \frac{\left[(x-h) \sin{\phi}-(y-k) \cos{\phi}\right]^2}{b^2} = 1,
\]
where $\phi$ is the counter-clockwise angle from the positive x-axis to the
semi-major axis $a$, $b$ is the semi-minor axis, and $h, k$ are the x- and
y-coordinates, respectively, of ellipse's center. However, because ellipses
are a special case of conics---of which the circle, ellipse, parabola, or
hyperbola are members---they can also be represented as such, using the quadric
form,
\[
Ax^2 + Bxy + Cy^2 + Dx + Ey + F = 0.
\]
Now, if we furthermore convert this, the quadric form, into its matrix
equivalent,
\[
E = \begin{bmatrix}
      A   & B/2 & D/2 \\
      B/2 & C   & E/2 \\
      D/2 & E/2 & F
    \end{bmatrix},
\]
we have arrived at the representation of an ellipse in projective geometry
required to find intersections between ellipses. Following this, we then
\begin{enumerate}
\item form a degenerate conic from the solution to the system consisting of the
  two conics we wish to intersect,
\item split this degenerate conic into a pencil of two lines, and finally
\item intersect the remaining conic with this pencil, yielding 0 to 4
  intersection points points (\cref{fig:intersection}).
\end{enumerate}

\begin{marginfigure}
\begin{knitrout}\small
\definecolor{shadecolor}{rgb}{0.969, 0.969, 0.969}\color{fgcolor}

{\centering \includegraphics[width=\maxwidth]{figure/graphics-intersection-1} 
\includegraphics[width=\maxwidth]{figure/graphics-intersection-2} 
\includegraphics[width=\maxwidth]{figure/graphics-intersection-3} 

}



\end{knitrout}
\caption{The process (from top to bottom) used to intersect two ellipses, here
yielding four points.}
\label{fig:intersection}
\end{marginfigure}

After we have all the intersection points, we find the overlap by examining the
intersection points that are formed from the intersections of the ellipses we
are currently exploring and that are simultaneously contained within all of
these ellipses. These points form a geometric shape that can be considered a
polygon with elliptical segments formed by successive points of the
polygon~(\cref{fig:polyarea}).

\begin{figure}[hbt]
\sidecaption{The overlap area between three ellipses is the sum of a convex polygon (in \textcolor{Grey}{grey}) and 2--3 ellipse segments (in \textcolor{SteelBlue4}{blue}).}
\begin{knitrout}\small
\definecolor{shadecolor}{rgb}{0.969, 0.969, 0.969}\color{fgcolor}

{\centering \includegraphics[width=\maxwidth]{figure/graphics-polyarea-1} 

}



\end{knitrout}
\label{fig:polyarea}
\end{figure}

Since the polygon part is always convex, it is easy to find its area using the
\emph{triangle method}. To find the areas of the elliptical segments, we first
order all the points in clockwise order\sidenote{It makes no difference if we
sort them in counter-clockwise order instead.}. Then, we acknowledge that each
elliptical segment is formed from an arc of the ellipse that is shared by both
points\sidenote{Because there is sometimes two arcs connecting the pairs of
points, we simply compute both areas and pick the smaller.}. Now that we have
two points on an ellipse, we can find the area of the ellipse segment using an
algorithm from \citet{eberly_area_2016}. To proceed, we

\begin{enumerate}
\item center their ellipse at $(0, 0)$,
\item normalize its rotation, which is not needed to compute the area,
\item integrate the ellipse from $0$ to $\phi_0$ and $\phi_1$ to produce two
  elliptical sectors,
\item subtract the smaller of these sectors from the larger, and
\item subtract the triangle section to finally find the segment
  area~\eqref{eq:segmentArea}.
\end{enumerate}

\begin{equation*}
\alpha(\theta_0, \theta_1) = F(\theta_1) - F(\theta_0) -
\frac{1}{2}\left|x_1y_0 - x_0y_1\right|,
\label{eq:segmentArea}
\end{equation*}
\[
\text{where } F(\theta) = \frac{a}{b}\left[ \theta -
\arctan{\left(\frac{(b - a)\sin{2\theta}}{b + a +(b - a )\cos{2\theta}} \right)}
\right]
\]
This procedure is illustrated in~\cref{fig:ellipsesegment}.

\begin{marginfigure}
\begin{knitrout}\small
\definecolor{shadecolor}{rgb}{0.969, 0.969, 0.969}\color{fgcolor}

{\centering \includegraphics[width=\maxwidth]{figure/graphics-ellipsesegment-1} 

}



\end{knitrout}
\caption{The elliptical segment in \textcolor{SteelBlue4}{blue} is found by
first subtracting the elliptical sector from $(a, 0)$ to $\theta_0$ from the one
from $(a, 0)$ to $\theta_1$ and then subtracting the triangle part
(in \textcolor{Grey}{grey}).}
\label{fig:ellipsesegment}
\end{marginfigure}

In a few instances\sidenote{1 out of approximately 7000 in our simulations},
the exact algorithm will break down. This occurs from numerical approximation
errors when some ellipses are close-to tangent to one another or completely
overlap. In these cases, the algorithm will resort to approximation of the
involved overlap area by
\begin{enumerate}
\item spreading points across the ellipses using Vogel's
  method~(see \nameref{sec:labeling} for an overview),
\item identifying the points that are inside the intersection via the inequality
  \begin{equation*}
  \begin{multlined}
  \frac{\left[ (x-h)\cos{\phi}+(y-k)\sin{\phi} \right]^2}{a^2} + \\
    \frac{\left[(x-h) \sin{\phi}-(y-k)\cos{\phi}\right]^2}{b^2} < 1,
  \end{multlined}
  \end{equation*}
  where $x$ and $y$ are the coordinates of the sampled points, and finally
\item approximating the area by multiplying the proportion of points inside the
  overlap with the area of the ellipse.
\end{enumerate}

With this in place, we are now able to compute the areas of all intersections
and their relative complements up to numerical precision. We feed the initial
layout computed in~\nameref{sec:initConfig} to the optimizer, this time allowing
the ellipses to rotate and the relation between the semiaxes vary, altogether
rendering five parameters to optimize per set and ellipse. For each iteration of
the optimizer, the areas of all intersections are analyzed and a measure of loss
returned. The loss we use is the sum of squared errors between the ideal sizes
($\omega$ from~\cref{def:omega}) and the respective areas of the diagram,
\begin{equation}
\sum_{i=1}^{n}  (A_i-\omega_i)^2
\label{eq:loss}
\end{equation}

\section{Goodness of fit}
\label{sec:gof}

When \pkg{eulerr} cannot find a perfect solution it offers an approximate one
instead, the adequacy of which has to be measured in a standardized way. For
this purpose we adopt two measures: \emph{stress}~\citep{wilkinson_2012} and
\emph{diagError}~\citep{micallef_2014}.

The stress metric is the residual sums of squares over the total sums of
squares,
\begin{equation}
\text{Stress} = \frac{\sum_{i=1}^n (\omega_i - A_i)^2}{\sum_{i=1}^n
  (A_i - \bar{A})},
\label{eq:stress}
\end{equation}
where $\bar{A}$ is the arithmetic mean of the areas in the diagram.

The stress metric does not lend itself readily to a clear-cut interpretation but
can be transformed into a rough analogue of the correlation coefficient by $r = \sqrt{1-\text{Stress}^2}$.

diagError, meanwhile, is given by
\begin{equation}
\text{diagError} = \max_{i = 1, 2, \dots, n} \left|
  \frac{\omega_i}{\sum_{i=1}^n \omega_i} -\frac{A_i}{\sum_{i=1}^n A_i} \right|,
\label{eq:diagError}
\end{equation}
which is the maximum absolute difference of the proportion of any $\omega$ to
the respective unique area of the diagram.

\chapter{Results}
\label{ch:results}

The only R packages that feature area-proportional Euler diagrams are
\pkg{eulerr}, \pkg{venneuler}, \pkg{Vennerable}, and \pkg{d3VennR}. The latter
is an interface to the \pkg{venn.js} script that has been discussed previously,
but because it features an outdated version of the script and only produces
images as html, we call \pkg{venn.js} directly using the \proglang{javascript}
engine \pkg{V8} via the R package of the same name.
Only \pkg{eulerr}, \pkg{venn.js}, and \pkg{venneuler} support more than three
sets, which is why there are only three-set results for \pkg{Vennerable} and
\pkg{eulerAPE}.

The packages used here were
\begin{itemize}
  \item \pkg{eulerr} 2.0.0.9000,
  \item \pkg{eulerAPE} 3.0.0,
  \item \pkg{venn.js} 0.2.14,
  \item \pkg{venneuler} 1.1-0, and
  \item \pkg{Vennerable} 3.1.0.9000.
\end{itemize}

The results in \nameref{sec:performance} and all results from \pkg{eulerAPE}
were computed on a computer\sidenote{%
  The specification of the computer was
    \begin{itemize}
      \item Microsoft Windows Pro 10 x64
      \item Intel\textregistered~Core\textsuperscript{TM} i7-4500U CPU @
            1.80GHz, 2 cores
      \item 8 Gb memory
    \end{itemize}
}
running R version~3.4.2. The remaining results were computed on an Amazon EC2
cloud-based computing instance.

\section{Case studies}
\label{sec:caseStudies}

We begin our examination of \pkg{eulerr} by studying a difficult set
relationship from \citet{wilkinson_2012},
\begin{gather*}
A = 4 \quad B = 6 \quad C = 3 \quad D = 2 \quad E = 7 \quad F = 3\\
A\& B = 2 \quad A\&F = 2 \quad B\& C = 2 \quad B\&D = 1 \\
B\& F = 2 \quad C\&D = 1 \quad D\& E = 1 \quad E\&F = 1 \\
A\&B\&F = 1 \quad B\&C\&D = 1,\end{gather*}
where we use the $\&$ operator as defined in~\cref{def:omega}. We fit this
specification with \pkg{venneuler} and \pkg{eulerr}, in the latter case using
both circles and ellipses~(\cref{fig:venneulerHard}).

This example showcases the improvement gained from using ellipses and also the
small benefit that \pkg{eulerr} offers relative to \pkg{venneuler}.

\begin{figure*}[thb]
\begin{knitrout}\small
\definecolor{shadecolor}{rgb}{0.969, 0.969, 0.969}\color{fgcolor}

{\centering \includegraphics[width=\maxwidth]{figure/graphics-venneulerHard-1} 

}



\end{knitrout}
\caption{A comparison of a Euler diagram generated with \pkg{venneuler} with two
generated from \pkg{eulerr} with circles and ellipses respectively. The stress
of the solutions are 0.006, 0.004, and 0.000 respectively.}
\label{fig:venneulerHard}
\end{figure*}

\section{Consistency}
\label{sec:consistency}

To compare the consistency among \pkg{eulerr}, \pkg{venneuler}, \pkg{eulerAPE},
\pkg{venn.js}, and \pkg{Vennerable}, we generate random diagrams of circles and
ellipses, compute their areas, and attempt to reproduce the original diagram
using the software packages that we are studying. We restrict ourselves to
diagrams consisting of between 3 and 8 shapes.

For the circles, we sample radii ($r_i$) and coordinates ($h_i$ and $k_i$) from
%
\begin{equation}
\begin{aligned}
r_i     & \sim \mathcal{U}(0.3, 0.6)\\
h_i,k_i & \sim \mathcal{U}(0, N/3),
\end{aligned}
\label{eq:consistencyCircles}
\end{equation}
where $N$ is the number of shapes.
For the ellipses, we sample semiaxes ($a_i$ and $b_i$), coordinates
($h_i$ and $k_i$), and rotation axes ($\phi_i$) from
%
\begin{equation}
\begin{aligned}
h_i,k_i & \sim \mathcal{U}(0, i/3)\\
r_i     & \sim \mathcal{U}(0.3, 0.6)\\
c_i     & \sim \mathcal{U}(1/3, 1)\\
a_i     & = r_ic_i\\
b_i     & = r_i/c_i\\
\phi_i  & \sim \mathcal{U}(0, 2\pi),
\end{aligned}
\label{eq:consistencyEllipses}
\end{equation}
where $c$ is the

Next, we compute the required areas, $\omega$ (from~\cref{def:omega}), for each
iteration and fit a Euler diagram using the aforementioned packages. Finally,
we compute and return \emph{diagError}~\eqref{eq:diagError} and score each
diagram as a \emph{success} if its \emph{diagError} is lower than 0.01, that is,
if no portion of the diagram is 1\% off (in absolute terms) from that of the
input; note that this is always achievable since our Euler diagrams are formed
from sampled diagrams.

For each number of shapes $(i=3,4,\dots,8)$ we run the simulations until we have achieved a 95\% confidence interval around $p$, the proportion of successful diagrams, that is no wider than 2\%. The confidence that we generate is the standard asymptotic interval,
\begin{equation*}
\mathcal{I}(p)_{0.95} = p \pm z_{0.95}\sqrt{\frac{p(p-1)}{n}}
\end{equation*}
The algorithm is formalized in~\autoref{alg:consistency}. The left panel of
\cref{fig:consistency} shows the results of our simulation.

\begin{alg}
\DontPrintSemicolon
\sidecaption{The algorithm used to simulate circles and ellipses, compute their
areas, and fit Euler diagrams to these layouts using the different software packages.\label{alg:consistency}}
\For{$i\leftarrow 3$ \KwTo $8$}{
  \While{length of each $\mathcal{I}(p_s)_{0.95} < 2\%$ \textbf{and} $j < 500$}{
    \uIf{circle}{
      $h,k    \leftarrow \mathcal{U}(0, 1)$\\
      $r      \leftarrow \mathcal{U}(0.3, 0.6)$\\
      $\omega \leftarrow \mathtt{findOverlaps}(h, k, r)$\\
    }
    \ElseIf{ellipse}{
      $h,k    \leftarrow \mathcal{U}(0, 1)$\\
      $a,b    \leftarrow \mathcal{U}(0.2, 0.8)$\\
      $\phi   \leftarrow \mathcal{U}(0, 2\pi)$\\
      $\omega \leftarrow \mathtt{findOverlaps}(h, k, a, b, \phi)$\\
    }
    $A \leftarrow \mathtt{ fitDiagram}(\omega)$\\
    $\text{diagError}_j \leftarrow \max_{k=1,2,\dots,2^i-1} \left| \frac{\omega_k}{\sum \omega_k} - \frac{A_k}{\sum A_k} \right|$\\
    \lIf{$\text{diagError}_j < 0.01$}{$\text{successes} + 1$}
    \ForEach{$s \leftarrow$ software package}{
      $\hat{p}_s \leftarrow \text{successes}/j$\\
      $\mathcal{I}(\hat{p}_s)_{0.95} \leftarrow \hat{p} \pm z_{0.95}\sqrt{\frac{\hat{p}_s(\hat{p}_s-1)}{n}}$\\
    }
  }
}
\end{alg}

\begin{figure*}[bhtp]

\begin{knitrout}\small
\definecolor{shadecolor}{rgb}{0.969, 0.969, 0.969}\color{fgcolor}

{\centering \includegraphics[width=\maxwidth]{figure/graphics-consistency-1} 

}



\end{knitrout}
\caption{Reproducibility tests for ellipses and circles generated from the
distributions from \eqref{eq:consistencyCircles}. Note that \pkg{Vennerable}
and \pkg{eulerAPE} only support Euler diagrams for three sets, which is why its
data is absent for the other cases.\label{fig:consistency}}
\end{figure*}



\pkg{eulerr} outperforms both \pkg{Vennerable} and \pkg{venneuler} in
consistency~(\cref{fig:consistency}). It is able to reproduce Euler diagrams for
almost all of the circles and for all three-set ellipses. For ellipses of four
or more sets, the consistency drops considerably, yet remains above
\ensuremath{\infty{}}\%. \pkg{Vennerable}, which
is only able to produce three-set diagrams, only produces accurate diagrams for
\ensuremath{NaN}\% of the random layouts and moreover
fails with an error in 0 cases.

\section{Accuracy}
\label{sec:accuracy}

In \nameref{sec:consistency}, we assess the efficacy in reproducing diagrams
with exact, but unknown, solutions. In real situations, however, we are often
faced with set configurations that lack exact solutions, in which case we want
our method to produce an approximation that is as accurate as possible.

To assess this, we generate random set relationships, that may or may not have
exact solutions, and fit Euler diagram using the software under study. For each
$i=3,4,\dots,8$ sets we initialize $2^i-1$ permutations of set combinations,
select one for each set, and initialize these to a number in
$\mathcal{U}(0, 1)$. After this, we pick $0 \text{ to } \binom{N-2}{1}$ elements
from the $2^i-i-1$ remaining permutations and assign to them a number from
$\mathcal{U}(0, 1)$ as before.

As in \nameref{sec:consistency}, we run our simulations until we have achieved
a desired confidence interval for the estimates. This time, we keep on going
until we have a 95\% confidence interval no longer than 0.02 in mean diagError.
We use the common confidence level based on the t-distribution,
\begin{equation}
\mathcal{I}(\hat{\mu})_{0.95} = \hat{\mu} \pm t_{0.95}\frac{s}{\sqrt{n}},
\end{equation}
where $\hat{\mu} = \bar{x}$. The algorithm is formalized in~\autoref{alg:accuracy}

\begin{alg}
\DontPrintSemicolon
\sidecaption{The algorithm we use to simulate random set relationships and fit
them with the software under study to assess their accuracy.\label{alg:accuracy}}
\For{$i\leftarrow 3$ \KwTo $8$}{
  \While{length of each $\mathcal{I}(\hat{\mu}_s)_{0.95} < 0.02$ \textbf{and} $j < 500$}{
    initialize $\omega = \{\omega_1,\omega_2,\dots,\omega_{2^i-1} \}$ to zero\\
    \For{$j\leftarrow 3$ \KwTo $i$}{
      $j \leftarrow$ random index in $\{\omega : \omega \cap F_i \neq \emptyset\}$\\
      $\omega_j \leftarrow \mathcal{U}(0, 1)$\\
    }
    $\omega_S \leftarrow \mathcal{U}\{0, i\}$ random elements from $\{\omega : \omega = 0\}$\\
    $\omega_S \leftarrow \mathcal{U}(0, 1)$\\
    fit a Euler diagram to $\omega$\\
  }
}
\end{alg}

The error in the diagrams generated with \pkg{eulerr} is considerably lower
than for the other methods. This is most apparent with ellipses but also true
for circles compared to \pkg{venneuler} and \pkg{Vennerable}.

\begin{figure*}[hbt]
\begin{knitrout}\small
\definecolor{shadecolor}{rgb}{0.969, 0.969, 0.969}\color{fgcolor}

{\centering \includegraphics[width=\maxwidth]{figure/graphics-accuracy-1} 

}



\end{knitrout}
\caption{Accuracy tests of set relationships that may or may not have perfect
solutions, generated from~\eqref{eq:consistencyEllipses}.}
\label{fig:accuracy}
\end{figure*}

\section{Performance}
\label{sec:performance}

Using the same method as in \nameref{sec:accuracy}, we generate random set
relationships and measure the time it takes for each software package to form a
diagram from the fit. We rely on \pkg{microbenchmark} to compute this,
subtracting function call times to produce strict measurement. In addition, we
randomize the order in which the packages are called between trials.

\begin{figure*}[hbt]
\begin{knitrout}\small
\definecolor{shadecolor}{rgb}{0.969, 0.969, 0.969}\color{fgcolor}

{\centering \includegraphics[width=\maxwidth]{figure/graphics-performance-1} 

}



\end{knitrout}
\caption{Performance of \pkg{eulerr}, \pkg{venneuler}, and \pkg{Vennerable} on
random set relationships of 3 to 8 sets }
\label{fig:performance}
\end{figure*}

\chapter{Discussion}
\label{sec:discussion}

In this paper, we have presented a novel method for generating Euler diagrams
for any number of sets using ellipses. We have shown that the method is superior
in both accuracy and consistency next to all other software packages for
R---even when the method is restricted to circles. In addition, the method is
speedier than the competition for set relationships with up to six sets,
wherafter it performs worse than \pkg{venneuler}.

In terms of consistency and accuracy, there are several reasons for why
\pkg{eulerr} improves upon the method in \pkg{venneuler} and \pkg{Vennerable}.
The primary reason lies in the use of ellipses rather than circles. Ellipses
feature two additional degrees of freedom and are therefore able to accurately
represent a larger variety of relationships.

Moreover, the initial optimizer in \pkg{eulerr} circumvents a rule in
\pkg{venneuler} that puts unnecessary restrictions on layouts that include
disjoint or subset relationships. The initial optimizer in venneuler places
disjoint and subset circles exactly neck-in-neck and at the exact midpoint of
the set respectively. Yet, because we are indifferent about where in the space
outside (or respectively inside) the circles are placed, that behavior becomes
problematic since it might interfere with locations of other sets that need to
use that space. The MDS algorithm from \pkg{venn.js} circumvents this by
assigning a loss and gradient of zero when the pairwise set intersection
\emph{and} the candidate circles are disjoint or subset.

For 3 to 7 sets, \pkg{eulerr} is speedier than both \pkg{Vennerable} and
\pkg{venneuler}. The reasons for this is partly to do with the implementation
in C++ using high-performance interfaces such as
\pkg{Rcpp}~\citep{eddelbuettel_2011} and
\pkg{RcppArmadillo}~\citep{eddelbuettel_2014}. For few sets, the exact-area
calculations that \pkg{eulerr} features also promote better performance; yet,
paradoxically, this also happens to be the reason for why the performance of
\pkg{eulerr} suffers as the number of sets surge. The bottleneck is the final
optimizer. It has to examine every possible intersection when computing the
areas, thus converging in $\mathcal{O}(2^n)$ time. \citet{wilkinson_2012},
meanwhile, report convergence in $\mathcal{O}(n)$ time. This is clearly
evidenced in \cref{fig:performance}. Future versions of this algorithm might
consider implementing approximate area-calculations when the number of sets is
large.

This method was first published for circles in a blog
post~\citep{frederickson_2013} and in a scholarly paper for up to three
ellipses~\citep{micallef_2013} but has to our knowledge not previously been
generalized to any number of ellipses.

The authors motivate this limitation by the propensity of Euler diagrams with
more sets to lack adequate solutions and that their complexity make
implementations difficult~\citep{micallef_2013}.
\begin{fullwidth}
\part*{Appendices}
\end{fullwidth}
\appendix
\chapter{Visualization}

Once we have ascertained that our Euler diagram fits well, we can turn to
visualizing the solution. For this purpose, \pkg{eulerr} leverages the
\pkg{Lattice} graphics system~\citep{sarkar_2008} for R to offer intuitive and
granular control over the output.

Plotting the ellipses is straightforward using the parametrization of a rotated
ellipse,
%
\begin{equation*}
\begin{bmatrix}
  x \\ y
\end{bmatrix} =
\begin{bmatrix}
  h + a \cos{\theta} \\
  k + b \sin{\theta}
\end{bmatrix},\quad \text{where } \theta \in [0, 2\pi].
\end{equation*}
%
Often, however, we would also like to label the ellipses and their intersections
with text and this is considerably more involved.

\section{Labeling}
\label{sec:labeling}

Labeling the ellipses is difficult because the shapes of the intersections
often are irregular, lacking a well-defined center; we know of no analytical
solution to this problem. As usual, however, the next-best option turns out to
be a numerical one. First, we locate a point that is inside the required region
by spreading points across the discs involved in the set intersection. To
distribute the points, we use a modification of
\emph{Vogel's method}~\citep{arthur_2015,vogel_1979} adapted to ellipses.
Vogel's method spreads points across a disc using
\begin{equation}
p_k =
\begin{bmatrix}
  \rho_k \\
  \theta_k
\end{bmatrix} =
\begin{bmatrix}
  r \sqrt{\frac{k}{n}}\\
  \pi (3 - \sqrt{5})(k - 1)
\end{bmatrix}\quad\text{for } k = 1, 2,\dots, n.
\label{eq:vogel}
\end{equation}
In our modification, we scale, rotate, and translate the points formed
in~\eqref{eq:vogel} to match the candidate ellipse. We rely, as before, on
projective geometry to carry out the transformations in one go:
\[
p' =
\begin{bmatrix}
  x' \\
  y' \\
  1
\end{bmatrix} =
\begin{bmatrix}
  1 & 0 & h \\
  0 & 1 & k \\
  0 & 0 & 1
\end{bmatrix}
\begin{bmatrix}
  \cos{\phi}  & \sin{\phi} & 0 \\
  -\sin{\phi} & \cos{\phi} & 0\\
  0           & 0          & 1
\end{bmatrix}
\begin{bmatrix}
  a & 0 & 0 \\
  0 & b & 0 \\
  0 & 0 & 1
\end{bmatrix}
\begin{bmatrix}
  \hat{x} \\
  \hat{y} \\
  1
\end{bmatrix}
\]
After we spread our points throughout the ellipse and find a point, $p'_i$, that
is contained in our desired intersection, we proceed to optimize its position
numerically. The position we are looking for is that which maximizes the
distance to the closest ellipse in our diagram to provide as much margin as
possible for the label. This is a maximin problem with a loss function equal to
\begin{equation}
\max_{x,y \in \mathbb{R}^2} \min_{i=1,2,\dots,N} f(x,y,h_i,k_i,a_i,b_i,\phi_i)
\label{eq:lossDist}
\end{equation}
where $f$ is the function that determines the distance from a point ($x,y$) to the ellipse defined by $h,k,a,b$ and $\phi$.

Similarly to fitting Euler diagrams in the general case, there appears to be no
analytical solution tocomputig the distance from a point to an ellipses. The
numerical solution we use has been described in [ref] and involves a bisection
optimizer.

To optimize this, we employ a version of the
\emph{Nelder--Mead Method}~\citep{nelder_1965} which has been translated from
\citet{kelley_1999} and adapted for \pkg{eulerr} to ensure that quick
convergence and that the simplex remains within the intersection boundaries
(since we want the local maximum). The method is visualized in~\cref{fig:vogel}.
\begin{marginfigure}
\begin{knitrout}\small
\definecolor{shadecolor}{rgb}{0.969, 0.969, 0.969}\color{fgcolor}

{\centering \includegraphics[width=\maxwidth]{figure/graphics-vogel-1} 
\includegraphics[width=\maxwidth]{figure/graphics-vogel-2} 
\includegraphics[width=\maxwidth]{figure/graphics-vogel-3} 

}



\end{knitrout}
\caption{The method eulerr uses to locate an optimal position for a label in
three steps from top to bottom: first, we spread sample points on one of the
ellipses and pick one inside the intersection of interest, then we begin moving
it numerically, and finally place our label.}
\label{fig:vogel}
\end{marginfigure}

\section{Aesthetics}
\label{sec:aesthetics}

Euler diagrams display both quantitative and qualitative data. The quantitative
aspect is the quantities or sizes of the sets depicted in the diagram and is
visualized by the relative sizes, and possibly the labels, of the areas of the
shapes---this is the main focus of this paper. The qualitative aspects,
meanwhile, consist of the mapping of each set to some quality or category, such
as having a certain gene or not. In the diagram, these qualities can be
separated through any of the following aesthetics:
%
\begin{itemize}
\item color,
\item border type,
\item text labelling,
\item transperancy,
\item patterns,
\end{itemize}
%
or a combination of these. The main purpose of these aethetics is to separate
out the different ellipses so that the audience may interpret the diagram with
ease and clarity.

Among these aesthetics, the best choice (from a viewer perspective) appears to
be color~\citep{blake_2016}, which provides useful information without
extraneous chart junk~\citep{tufte_2001}. The issue with color, however, is that
it cannot be perceived perfectly by roughly 10\% of all people. Moreover, color
is often printed at a premium in scientific publications and adds nothing to a
diagram of two shapes.

For these reasons, \pkg{eulerr} defaults to distinguishing ellipses with color
using a color palette generated via the R package
\pkg{qualpalr}~\citep{larsson_2016}, which automatically generates qualitative
color palettes based on a perceptual model of color vision that optionally
caters to color vision deficiency. This palette has been manually modified
slightly to fullfil our other objectives of avoiding using colors for two sets.

\begin{figure}
\sidecaption{The eight first colors of the default color palette.}
\begin{knitrout}\small
\definecolor{shadecolor}{rgb}{0.969, 0.969, 0.969}\color{fgcolor}

{\centering \includegraphics[width=\maxwidth]{figure/graphics-colorexamle-1} 

}



\end{knitrout}
\label{fig:colorexample}
\end{figure}

\section{Normalizing dispered layouts}
\label{sec:layout}

A side effect of running an unconstrained optimizer (See \nameref{sec:finalConfig}) is that we almost invariably produce overdispersed layouts if there are disjoint clusters of ellipses. To solve this, we use a SKYLINE-BL rectangle packing algorithm~\citep{jylanki_2010} which is designed specifically for \pkg{eulerr}. In it, we surround each ellipse cluster with a bounding box and pack these boxes into a bin with an appropriate size and aspect ratio of the golden rule.

\chapter{Usage}\label{ch:usage}

\code{euler()} and \code{plot()} are the only functions that a user of
\pkg{eulerr} need concern themselves with. In \nameref{sec:input}, we described
the various forms of input that the function can be supplied with. Using the
first form, we may input a diagram from \citet{junta_2009} that was showcased in
\citet{wilkinson_2012} as follows.

\begin{knitrout}\small
\definecolor{shadecolor}{rgb}{0.969, 0.969, 0.969}\color{fgcolor}\begin{kframe}
\begin{alltt}
\hlkwd{library}\hlstd{(eulerr)}
\hlstd{junta_2009} \hlkwb{<-} \hlkwd{c}\hlstd{(}\hlstr{"SE"} \hlstd{=} \hlnum{13}\hlstd{,} \hlstr{"Treat"} \hlstd{=} \hlnum{28}\hlstd{,} \hlstr{"Anti-CCP"} \hlstd{=} \hlnum{101}\hlstd{,}
                \hlstr{"DAS28"} \hlstd{=} \hlnum{91}\hlstd{,} \hlstr{"SE&Treat"} \hlstd{=} \hlnum{1}\hlstd{,} \hlstr{"SE&DAS28"} \hlstd{=} \hlnum{14}\hlstd{,}
                \hlstr{"Treat&Anti-CCP"} \hlstd{=} \hlnum{6}\hlstd{,} \hlstr{"SE&Anti-CCP&DAS28"} \hlstd{=} \hlnum{1}\hlstd{)}
\hlstd{fit1} \hlkwb{<-} \hlkwd{euler}\hlstd{(junta_2009)}
\end{alltt}
\end{kframe}
\end{knitrout}

Printing the results provides a summary of the fit, including the stress and
diagError metrics that were introduced in~\nameref{sec:gof}.

\begin{knitrout}\small
\definecolor{shadecolor}{rgb}{0.969, 0.969, 0.969}\color{fgcolor}\begin{kframe}
\begin{alltt}
\hlstd{fit1} \hlcom{# or equivalently print(fit1)}
\end{alltt}
\begin{verbatim}
##                         original fitted residuals regionError
## SE                            13     13         0       0.000
## Treat                         28     28         0       0.000
## Anti-CCP                     101    101         0       0.002
## DAS28                         91     91         0       0.001
## SE&Treat                       1      1         0       0.000
## SE&Anti-CCP                    0      0         0       0.000
## SE&DAS28                      14     14         0       0.000
## Treat&Anti-CCP                 6      6         0       0.000
## Treat&DAS28                    0      0         0       0.000
## Anti-CCP&DAS28                 0      0         0       0.000
## SE&Treat&Anti-CCP              0      0         0       0.000
## SE&Treat&DAS28                 0      0         0       0.000
## SE&Anti-CCP&DAS28              1      0         1       0.004
## Treat&Anti-CCP&DAS28           0      0         0       0.000
## SE&Treat&Anti-CCP&DAS28        0      0         0       0.000
## 
## diagError: 0.004 
## stress:    0
\end{verbatim}
\end{kframe}
\end{knitrout}

The fit is more or less equivalent to that of
\pkg{venneuler}~\citep{wilkinson_2012}. We could try to fit the diagram using
ellipses instead.

\begin{knitrout}\small
\definecolor{shadecolor}{rgb}{0.969, 0.969, 0.969}\color{fgcolor}\begin{kframe}
\begin{alltt}
\hlcom{# Fit the data using ellipses instead}
\hlstd{fit2} \hlkwb{<-} \hlkwd{euler}\hlstd{(junta_2009,} \hlkwc{shape} \hlstd{=} \hlstr{"ellipse"}\hlstd{)}

\hlcom{# Compare the fits on diagerror}
\hlstd{fit1}\hlopt{$}\hlstd{diagError} \hlopt{-} \hlstd{fit2}\hlopt{$}\hlstd{diagError}
\end{alltt}
\begin{verbatim}
## [1] 0
\end{verbatim}
\end{kframe}
\end{knitrout}

Comparing the two fits in diagError, however, shows that we have not bettered
the fit in any meaningful way. Our next goal is to visualize the layout, which
we do both using the default options and by customizing the fit, adding
counts, replacing the sets' labels with a key, removing lines, and changing the
fills using the \pkg{RColorBrewer} package~(\cref{fig:plotting}).

\begin{knitrout}\small
\definecolor{shadecolor}{rgb}{0.969, 0.969, 0.969}\color{fgcolor}\begin{kframe}
\begin{alltt}
\hlstd{p1} \hlkwb{<-} \hlkwd{plot}\hlstd{(fit1)}
\hlstd{p2} \hlkwb{<-} \hlkwd{plot}\hlstd{(fit1,}
           \hlkwc{counts} \hlstd{=} \hlkwd{list}\hlstd{(}\hlkwc{fontface} \hlstd{=} \hlnum{3}\hlstd{),}
           \hlkwc{fill} \hlstd{= RColorBrewer}\hlopt{::}\hlkwd{brewer.pal}\hlstd{(}\hlnum{4}\hlstd{,} \hlstr{"Set2"}\hlstd{),}
           \hlkwc{border} \hlstd{=} \hlstr{"transparent"}\hlstd{,}
           \hlkwc{auto.key} \hlstd{=} \hlkwd{list}\hlstd{(}\hlkwc{space} \hlstd{=} \hlstr{"right"}\hlstd{))} \hlcom{# key on the right}
\end{alltt}
\end{kframe}
\end{knitrout}

\begin{figure}[hbtp]
\sidecaption{The same fit visualized in two distinct ways.\label{fig:plotting}}
\begin{minipage}[b]{.4\linewidth}
\begin{knitrout}\small
\definecolor{shadecolor}{rgb}{0.969, 0.969, 0.969}\color{fgcolor}

{\centering \includegraphics[width=\maxwidth]{figure/graphics-twofits1-1} 

}



\end{knitrout}
\subcaption{The default settings.}
\end{minipage}%
\begin{minipage}[b]{.6\linewidth}
\begin{knitrout}\small
\definecolor{shadecolor}{rgb}{0.969, 0.969, 0.969}\color{fgcolor}

{\centering \includegraphics[width=\maxwidth]{figure/graphics-twofits2-1} 

}



\end{knitrout}
\subcaption{Custom plot settings.}
\end{minipage}
\end{figure}

\begin{fullwidth}
\bibliography{eulerr}
\bibliographystyle{unsrtnat}
\end{fullwidth}

\end{document}
